\begin{abstracten}
Requirements for portfolio management is different among different investors. Risk is one of the critical factors. Max Drawdown (MDD), the maximum observed loss from a peak to a trough of a portfolio, is a commonly used risk indicator. Reinforcement Learning (RL) is a promising machine learning approach for portfolio selection. Instead of using investment return as a reward function directly, many models use indicators that have taken variability into account, like Sharpe Ratio \cite{Sharpe49} or Sterling ratio. However, neither Sharpe ratio nor Sterling ratio has inputs to raise or reduce risk's influence upon the reward function. Therefore the model using Sharpe Ratio \cite{Sharpe49} or Sterling ratio cannot reflect the needs of investor types with different tolerance to risk.

\par
In this thesis, we introduce a reward function for Portfolio Management RL model which includes the influence of MDD as its parameters.   

asset allocation reinforcement learning reward function


\noindent
Keywords: Reinforcement Learning, Max Drawdown, Portfolio Management,
\end{abstracten}
