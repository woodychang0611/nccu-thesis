\chapter{Introduction}

Based on investors' preferences, Modern Portfolio Theory (MPT) can produce a portfolio with maximum expected return from a given variance\cite{10.2307/2975974}. Reinforcement Learning (RL) is one of the popular approaches for advanced portfolio management systems.  However, despite many RL-based portfolio performances uses measures that are risk-adjusted\cite{cogneau2009101}, like the Shape ratio\cite{Sharpe49} or the Sterling ratio\cite{magdon2004maximum}. Unfortunately, these measures do not provide the parameter to represent investor's risk preferences. As a result,  few RL-based portfolio management systems incorporate investors' risk preferences like MPT. 

\section{Objective}
RL is a compelling and popular approach for driving portfolio management systems. But, unfortunately, it will be a challenge for these systems to provide optimal user experience to investors if the system optimizes with the same objective function.

We aim to design a DRL-based portfolio management system to build portfolios that fit investors better by considering their preferences, specifically risk preferences. Surveys or other techniques by professional personnel or organization will interpret investors' risk preference to comparable indices. Traditionally, an advisor will suggest a portfolio with fixed weights based on these indices.
Our system will start from the same indices and target two objectives.
\begin{enumerate}
    \item  Provide a parameter to incorporate investors' risk preferences.
    \item Outperform a constant rebalanced portfolio which represents the portfolio suggested by the advisor. 
\end{enumerate}


\section {Portfolio Management System}
Portfolio management is an approach to create a collection of investments, also known as a portfolio, based on the state of the market, the investment universe, and investors' preferences.

The portfolio management system aims to build an optimal portfolio, which does not merely maximize the return but maximizes the utility based on investors' preferences. Among these preferences, risk preference is one of the significant factors.

Traditionally portfolio management can be separated into two stages. The first stage produces the forecasts. The second stage builds the portfolio based on the forecasts.  Modern Portfolio Theory (MPT), introduced by  Harry Markowitz \cite{10.2307/2975974} focus on the second stage. Machine Learning (ML) or Reinforcement Learning (RL) can complete both stages in a single process and construct portfolios that out-performed the general market\cite{KRAUSS2017689, moody2001learning}

% \section{Risk Measure}
Risk is an abstract concept, and there are many indicators to represent the risk. We will discuss some of them here. 
\subsubsection{Standard Deviation}
We can measure the volatility of the return with the standard deviation. The well known Sharpe Ratio uses the standard deviation to represent a risk. The downside for the standard deviation of return is assets that produce a blooming return that most investors seek will also yield a high standard deviation.
\subsubsection{Maximum Drawdown}
Drawdown is the decline from the highest peak. Maximum drawdown (MDD) is the maximum decline over the given period and reflects the actual loss of the investors. The Sterling ratio uses MDD as an indicator of the risk. 
% \section{Risk-Adjusted Measures}
There are many portfolio performance measures.\cite{cogneau2009101,cogneau2009more,cogneau2009more2}
Risk-Adjusted measures incorporate risk into the measure of performance for finance investments and reflect investors' nature better than using return along.
We will discuss a few of them over here.
\subsubsection{Sharpe Ratio}
Sharpe Ratio\cite{Sharpe49} is one of the most well-known risk-adjusted measure. It represent amount of return per unit of variation, or risk.\\
The revision version is defined as
\[ SR = \frac{E(R_a - R_b)}{\sigma_a},
\sigma_a = \sqrt{VAR(R_a-R_b)}\]
where \(R_a\) is the return of the assert, 
\(R_b\) is the risk-free return,
\(E(R_a - R_b)\) is the expected excess return of the assert,
and \(\sigma_a\) is standard deviation of the excess return.
One downside of the Sharpe Ratio is that it penalizes occasional high returns\cite{9206647}.
\subsubsection{Sterling ratio}
Sterling ratio\cite{magdon2004maximum} uses max drawdown (MDD) over a given period to represent the risk; this resolves the issue Sharpe Ratio has with occasional high returns. 
There are many forms of Sterling ratio; one common definition\cite{magdon2004maximum} is defined as 
\[ SR = \frac{E(R_a - R_b)}{MDD - 10\%}\]
%\subsubsection{Calmar ratio}
%Calmar ratio\cite{young1991calmar} replace empirical max drawdown with the expected maximum drawdown \(E(MDD)\), defined as 
%\[C_T = \frac{E(R_a - R_b)}{E(MDD)}\]


\subsubsection{Other Measures of Performance}
There are many other Sharpe ratio variations, like Burke Ratio, Martin Ratio, or Pain Ratio\cite{bacon2009sharp}. Furthermore, Philippe Cogneau and Georges Hübner censused the 101 performance measures for portfolios. Many of the measures are risk-adjusted measures\cite{cogneau2009101,cogneau2009more,cogneau2009more2}.
\section{Supervised Learning Forecast System}
Supervised learning is a popular approach for forecast systems. Christopher Krauss successfully used various machine learning techniques, including deep neural networks, to create ensembles predicting the S\&P 500 index that out-performed the general market\cite{KRAUSS2017689}.
\par

Thomas Fischer brings deep learning to the next level by using Long Short-Term Memory (LSTM) network\cite{FISCHER2018654}. His model demonstrates LSTM can effectively extract meaningful information from noisy financial time series data and beat other machine learning models in Christopher Krauss's article\cite{KRAUSS2017689} for most situations, except the crisis in 2009, where Random Forest perform better than LSTM.
\section{RL-baesd Portfolio Management System}
One major issue of the supervised forecast system is the loss of information. Follow-up systems using the forecasts to build the portfolio do not use inputs for the forecast system.
\par
John Moody and Lizhong Wu introduced Reinforcement Learning (RL) for the trading system \cite{618952}. Trading systems trained via RL can incorporate market features with trading cost effects, e.g., transaction costs and taxes, and outperformed trading systems trained via supervised learning with labels. For utility function, they observed that the result maximizing the differential Sharpe ratio produced is more consistent than the result via maximizing profits\cite{618952}.
\par
Despite many RL-based portfolio performances uses measures that are risk-adjusted\cite{cogneau2009101}, like the Shape ratio\cite{Sharpe49} or the Sterling ratio\cite{magdon2004maximum}. Unfortunately, these measures do not incorporate investor's risk preferences via parameters. As a result, few RL-based portfolio management systems incorporate investors' risk preferences like MPT. 
\label{c:intro}

