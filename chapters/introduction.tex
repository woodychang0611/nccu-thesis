\chapter{Introduction}

Based on investors' preferences, Modern Portfolio Theory (MPT) can produce a portfolio with maximum expected return from a given variance\cite{10.2307/2975974}. Reinforcement Learning (RL) is one of the popular approaches for advanced portfolio management systems.  However, despite many RL-based portfolio performances uses measures that are risk-adjusted\cite{cogneau2009101}, like the Shape ratio\cite{Sharpe49} or the Sterling ratio\cite{magdon2004maximum}. Unfortunately, these measures do not provide the parameter to represent investor's risk preferences. As a result,  few RL-based portfolio management systems incorporate investors' risk preferences like MPT. 

\section{Objective}
RL is a compelling and popular approach for driving portfolio management systems. But, unfortunately, it will be a challenge for these systems to provide optimal user experience to investors if the system optimizes with the same objective function.

We aim to design a DRL-based portfolio management system to build portfolios that fit investors better by considering their preferences, specifically risk preferences. Surveys or other techniques by professional personnel or organization will interpret investors' risk preference to comparable indices. Traditionally, an advisor will suggest a portfolio with fixed weights based on these indices.
Our system will start from the same indices and target two objectives.
\begin{enumerate}
    \item  Provide a parameter to incorporate investors' risk preferences.
    \item Outperform a constant rebalanced portfolio which represents the portfolio suggested by the advisor. 
\end{enumerate}


\subsection {Portfolio Management System}
A portfolio is a collection of financial investments. The goal of portfolio selection is to construct an optimal portfolio, which is usually different between investors due to their characteristics. Among these characteristics, risk preference is one of the significant factors. There are many indicators to represent the risk of the portfolio. Maximum Drawdown (MDD),  maximum loss from a peak over a given period, is a commonly used indicator of the risk.

Traditionally portfolio selection can be divided into two stages. First stage constructs the belief of future performance of available financial products. The second stage starts from the first stage and produces the choice of portfolio.  Modern Portfolio Theory (MPT) introduced by  Harry Markowitz \cite{10.2307/2975974} focus on the second stage. Machine Learning (ML) or Reinforcement Learning (RL) can complete both stages in a single process and construct portfolios that out-performed the general market\cite{KRAUSS2017689}

% \subsection{Risk Measure}
\subsubsection{Standard Deviation}

\subsubsection{Drawdown}
\subsubsection{Value at Risk}
% \section{Risk Adjusted Measures of Performance}
There are many portfolio performance measures.\cite{cogneau2009101,cogneau2009more,cogneau2009more2}
Risk-Adjusted measures incorporate risk into the measure of performance for finance investments and reflect investors' nature better than using return along.
We will discuss a few of them over here.
\subsection{Sharpe Ratio}
Sharpe Ratio\cite{Sharpe49} is one of the most well-known risk-adjusted measure. It represent amount of return per unit of variation, or risk.\\
The revision version is defined as
\[ SR = \frac{E(R_a - R_b)}{\sigma_a},
\sigma_a = \sqrt{VAR(R_a-R_b)}\]
where \(R_a\) is the return of the assert, 
\(R_b\) is the risk-free return,
\(E(R_a - R_b)\) is the expected excess return of the assert,
and \(\sigma_a\) is standard deviation of the excess return.
One downside of the Sharpe Ratio is that it penalizes occasional high returns.\cite{9206647}
\subsection{Sterling ratio}
Sterling ratio\cite{magdon2004maximum} uses max drawdown (MDD) over a given period to represent the risk; this resolves the issue Sharpe Ratio has with occasional high returns. 
There are many forms of Sterling ratio; one common definition\cite{magdon2004maximum} is defined as 
\[ SR = \frac{E(R_a - R_b)}{MDD - 10\%}\]
\subsection{Calmar ratio}
Calmar ratio\cite{young1991calmar} replace empirical max drawdown with the expected maximum drawdown \(E(MDD)\), defined as 
\[C_T = \frac{E(R_a - R_b)}{E(MDD)}\]
With the assumption that the trading system follows Brownian motion where \(\mu\) is the average return per unit time, \(\tau\) is the standard deviation of the returns per unit time, and T is the interval of size, the expected maximum drawdown \(E(MDD)\) can be defined as\cite{1196267,pratap2004maximum}:
\[E(MDD)=
\begin{cases}
    \cfrac{\tau^2}{\mu}Q_p(\cfrac{\mu^2T}{\tau^2}),&\text{if  }\mu > 0\\
    1.2552 \tau \sqrt{T},& \text{if  }\mu=0 \\
    \cfrac{\tau^2}{\mu}Q_n(\cfrac{\mu^2T}{\tau^2}),&\text{if  }\mu <0
\end{cases}
\]
where
\[
\begin{aligned}
Q_p(x) = &2\int_0^\infty\Biggl[
    e^{-u}\sum_{n=1}^{\infty}
    \cfrac{\sin^3(\theta_n)}{\theta_n-\cos(\theta_n)\sin(\theta_n)}
    (1-e^{-\frac{x}{2cos^2(\theta_n)}})\\
    &+e^{-\frac{u}{\tanh(u)}}\sinh(u)
    (1-e^{-\frac{x}{2\cosh^2(\theta_n)}})
    \Biggr]\,du
\end{aligned}
\]
\[
Q_n(x)-2\int_0^\infty e^u \sum_{n=1}^{\infty}
\cfrac{\sin^3(\theta_n)}{\theta_n-\cos(\theta_n)\sin(\theta_n)}
(1-e^{-\frac{x}{2cos^2(\theta_n)}})
\,du
\]
\[
\theta_n \text{ is the solution of }
 \tanh(\theta_n) =\cfrac{\theta_n}{u},
 \theta_n \in \bigl( (n-\cfrac{\pi}{2}),(n+\cfrac{\pi}{2})\bigr)
\]

\subsection{Other Measures of Performance}
There are many other Sharpe ratio variations, like Burke Ratio, Martin Ratio, or Pain Ratio\cite{bacon2009sharp}. Furthermore, Philippe Cogneau and Georges Hübner censused the 101 performance measures for portfolios. Many of the measures are risk-adjusted measures.\cite{cogneau2009101,cogneau2009more,cogneau2009more2} We will not discuss them in detail.
\subsection{Supervised Learning Forecast System}
Supervised learning is a popular approach for forecast systems. Christopher Krauss successfully used various machine learning techniques, including deep neural networks, gradient-boosted-trees, and random forests, to create ensembles predicting the S\&P 500 index that out-performed the general market\cite{KRAUSS2017689}.
\par

Thomas Fischer brings deep learning to the next level by using Long Short-Term Memory (LSTM) network\cite{FISCHER2018654}. His model demonstrates LSTM can effectively extract meaningful information from noisy financial time series data and beat other machine learning models in Christopher Krauss's article\cite{KRAUSS2017689} for most situations, except the crisis in 2009, where Random Forest perform better than LSTM.
\section{RL baesd Portfolio Management System}
One major issue of the supervised forecast system is the loss of information. Follow-up systems using the forecasts to build the portfolio do not use inputs for the forecast system.
\par
John Moody and Lizhong Wu introduced Reinforcement Learning (RL) for the trading system \cite{618952}. Trading systems trained via RL can incorporate market features with trading cost effects, e.g., transaction costs and taxes, and outperformed trading systems trained via supervised learning with labels. For utility function, they observed that the result maximizing the differential Sharpe ratio produced is more consistent than the result via maximizing profits\cite{618952}.
\label{c:intro}

