\section{Modern Portfolio Theory (MPT)}
Harry Markowitz introduced modern portfolio theory (MPT) in 1952. Instead of maximizing the expected return, the objective of MPT is to find the maximum expected return on a given risk. The variance of the portfolio is used as an indicator of risk. \cite{10.2307/2975974}
\par
For an N assets portfolio, \(E(R_i)\) and  \(\sigma_i\) is the expected return and standard deviation of the asset \(i\). \(w_i\) is the weighting of the asset in the portfolio.\(\rho_{ij}\) is the correlation coefficient between the returns on assets i and j.
The expected return and variance of the portfolio, \(E(R_p)\) and \(\sigma_p^2\), are defined as:
\[ E(R_p) = \sum_i^N w_i E_i \]
\[\sigma_p^2 = \sum_i \sum_j w_i w_j \sigma_i \sigma_j \rho_{ij}\]
where
\[\forall w: w \geq 0 \quad \sum_i ^N w_i = 1\]
\par
The next step is plot expected return and variance of all portfolios and find the efficient frontier. Than we can identify the portfolio with the maximize return on a given risk, or, vice versa, the lowest risk on a given expected return from the efficient frontier.