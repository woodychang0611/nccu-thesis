\section{Comparison}
MPT uses historical data of individual investments to forecast future returns and risks. Portfolio meets investor's risk preferences is available by using the efficiency frontier.
\par
RL-based portfolio management systems can use alternative inputs from other sources, including indices representing short-term or longer-term situations. However, it is a challenge for RL-based portfolio management systems to produce portfolios that meet investor's risk preferences like MPT as the parameter to represent investor's risk preferences is missing. 
\par
We aim to close the gap by introducing a DRL portfolio management system with adjustable risk preferences.
\begin{table}[hbt]
    \centering
    \begin{tabular}{|| c|| c | c || }
    \hline \hline
    \vtop{\hbox{\strut Portfolio Management}\hbox{\strut System}}
     & Adjustable risk preferences & Alternative input sources \\     \hline \hline
    MPT & \color{blue}{Yes} & No \\  \hline
    RL with Sharpe/Sterling ratio& No & \color{blue}{Yes}  \\  \hline
    Proposed System & \color{blue}{Yes} & \color{blue}{Yes} \\   \hline \hline
    \end{tabular}
    \caption{Comparison of Portfolio Management System}
    \label{tab:comparison_portfolio_system}
\end{table}