\section{RL-baesd Portfolio Management System}
One major issue of the supervised forecast system is the loss of information. Follow-up systems using the forecasts to build the portfolio do not use inputs for the forecast system.
\par
John Moody and Lizhong Wu introduced Reinforcement Learning (RL) for the trading system \cite{618952}. Trading systems trained via RL can incorporate market features with trading cost effects, e.g., transaction costs and taxes, and outperformed trading systems trained via supervised learning with labels. For utility function, they observed that the result maximizing the differential Sharpe ratio produced is more consistent than the result via maximizing profits\cite{618952}.
\par
Despite many RL-based portfolio performances uses measures that are risk-adjusted\cite{cogneau2009101}, like the Shape ratio\cite{Sharpe49} or the Sterling ratio\cite{magdon2004maximum}. Unfortunately, these measures do not incorporate investor's risk preferences via parameters. As a result, few RL-based portfolio management systems incorporate investors' risk preferences like MPT. 